\section{Git Hosting Providers}
\headlineframe{Git Hosting Providers}

\begin{frame}[c]{Git Hosting Providers}
  \begin{itemize}
    \item Several Providers and self-hosted server solutions are available
    \item Usually provide much more than just hosting the repositories
      \begin{itemize}
        \item Issue tracking
        \item Code review using pull requests
        \item Wiki
        \item Project Management, e.g. Canban boards
        \item Continuous integration
        \item Releases
      \end{itemize}
  \end{itemize}
\end{frame}

\begin{frame}[c]{Git Hosting Providers}
  \small
  \begin{tabu}{@{} X[,C] @{} X[,C] @{} X[,C] @{}}
    \href{https://github.com}{\includegraphics[width=0.75\linewidth]{figures/github.png}} &
    \href{https://gitlab.com}{\includegraphics[width=0.75\linewidth]{figures/gitlab.png}} &
    \href{https://bitbucket.org}{\includegraphics[width=0.75\linewidth]{figures/bitbucket.png}} \\
    \begin{itemize}
      \item Largest Hoster
      \item Many Open Source Projects, e.g. Python
      \item Unlimited private repositories
      \item Free CI service for public repositories
      \item Github Pro free for students / teachers / researchers
    \end{itemize}
    &
    \begin{itemize}
      \item open-source community edition
      \item paid enterprise edition with more features
      \item unlimited private repositories
      \item Self hosted or as service at \href{https://gitlab.com}{gitlab.com}
    \end{itemize}
    &
    \begin{itemize}
      \item Unlimited private repos with up to 5 contributors
      \item Lacks far behind GitHub and GitLab
    \end{itemize}
  \end{tabu}
\end{frame}

\begin{frame}[fragile]{SSH Keys}
  Git can communicate using two ways with a remote:
  \begin{description}
    \item[HTTPS] Works out of the box but requires entering credentials at every push/pull
    \item[SSH] Using private/public keys. \\ Usually, you only need to decrypt a key once per session.
  \end{description}

  \begin{itemize}
    \item To use ssh, you need to add at least one public key to your profile. 
    \item It is considered best practice to use unique keys per machine and service
  \end{itemize}

  \begin{enumerate}
    \item Create the key using: (choose a new password when asked)
      \begin{code}[title={Works in Powershell (Windows) and Unix Systems}]{bash}
        $ ssh-keygen -t ed25519 -C "GitHub Key for <username> at <machine>" -f "$HOME/.ssh/id_ed25519.github"
      \end{code}
    \item Copy the content of the public key, on unix with \mintinline{bash}+cat ~/.ssh/id_rsa.github.pub+ or by using a text editor to open the file
    \item Add the public key to your profile
  \end{enumerate}
\end{frame}

\begin{frame}[c]{Pull / Merge Requests}
  \begin{itemize}
    \item Pull Requests (GitHub) / Merge Requests (GitLab) are a feature on top of git provided by several platforms
    \item Used to propose changes by pushing a new branch and then requesting it to be merged into the main branch
    \item Usually, projects using the GitHub Workflow only allow changes to the master branch via Pull Requests
    \item Pull Requests are used for Code Review, project maintainers and co-developers can look at your code and
      ask for changes
    \item Usually, a Continuous Integration (CI) system runs checks for the changes proposed in a Pull Request
  \end{itemize}
\end{frame}

\begin{frame}[c]{Code Reviews}
  \begin{itemize}
    \item Code reviews are among the most essential parts of software development
    \item Similar to the peer-review process in science
    \item Get feedback and advice for improvements
    \item Prevent easy-to-find mistakes
    \item Ensure quality, performance, documentation, clarity of the software
    \item Developers can learn from each other immensely during code reviews
    \item You should require code reviews for pull requests
  \end{itemize}
\end{frame}

\begin{frame}[c]{How should you review code?}
  \begin{itemize}
    \item Automatize as much as possible before the actual human review
      \begin{itemize}
        \item Static code checks
        \item Unit tests / CI
        \item Coverage
        \item Code style checks
      \end{itemize}
    \item Focus on (in order):
      \begin{itemize}
        \item[\color{positive}\faCheckSquare] Are enough unit tests there?
        \item[\color{positive}\faCheckSquare] Are code and tests clear / explained in comments / following best practices?
        \item[\color{positive}\faCheckSquare] Any obvious performance improvements?\footnote{\enquote{Premature optimization is the root of all evil} –  Donald Knuth}
        \item[\color{positive}\faCheckSquare] Is the code documented?
      \end{itemize}
    \item Stay friendly but be concise
  \end{itemize}
\end{frame}


\begin{frame}[c]{Forking}
  \begin{itemize}
    \item Using git and hosting providers, it's easy to contribute to projects you do not have write access to.

    \item This is arguably the most important reason for git's success.

    \item Forking means to create a copy of the main repository in your namespace, e.g. \url{http://github.com/matplotlib/matplotlib} to \texttt{http://github.com/maxnoe/matplotlib}

    \item You can then make changes and create a pull request in the main repository!

    \item To keep your fork up to date, you should add both your fork and the main repo as remotes.
  \end{itemize}
\end{frame}

\begin{frame}[fragile]{Forks}
  We'll use the school repository for this example

  \begin{itemize}
    \item Click the \enquote{Fork} button on GitHub
    \item Clone your fork
      \begin{code}{bash}
        $ git clone git@github.com:maxnoe/school2021
      \end{code}
    \item Add the main repository as second remote. The name \enquote{upstream} is convention.
      \begin{code}{bash}
        $ git remote add upstream git@github.com:escape2020/school2021
      \end{code}
    \item Download content also from upstream
      \begin{code}{bash}
        $ git fetch upstream
      \end{code}
  \end{itemize}
\end{frame}

\begin{frame}[c, fragile]{Making a Pull Request from a forked Repository}
  \begin{itemize}
    \item When starting the new branch, make sure to start from the up-to-date upstream main/master:
      \begin{code}{bash}
        $ git fetch upstream
        $ git switch -c new_branch upstream/master
      \end{code}
    \item Make changes and commit
    \item When pushing the branch, specify your fork (origin):
      \begin{code}{bash}
        $ git push -u origin new_branch
      \end{code}
    \item Go to GitHub or click on the link in the push message to open the Pull Request
  \end{itemize}
\end{frame}

\begin{frame}[c]{Issue Tracking}
  \begin{itemize}
    \item Issue Trackers are an important part of every software project
      \begin{itemize}
        \item Report bugs
        \item Feature requests
        \item Project planning
        \item Ask for help
      \end{itemize}
    \item Issues can be linked to commits and pull requests
  \end{itemize}
\end{frame}

\begin{frame}[c, fragile]{Commit Integration with Issue Tracking}
  Start working on fixing a bug, that was documented in issue 42.

  \begin{code}{text}
  $ git checkout -b fix_42

  ... do stuff to fix bug ...

  $ git add src/foo.cxx
  $ git commit -m "Fix segmentation fault when doing stuff, fixes #42" 
  $ git push -u origin fix_42
  \end{code}

  If this commit get's merged into master, issue 42 will automatically be closed.
\end{frame}

\begin{frame}[c]{Continuous Integration}
  \begin{itemize}
    \item Strictly interpreted, continous integration means integrating current work
      \enquote{often} into the main version
    \item Usually, this means running automated builds and checks on a dedicated server
    \item Ideally, these are run for each push event
    \item For git projects, checks for pull requests should run on the merged result, not the branch itself
    \item You should require passing CI system for Pull Requests
  \end{itemize}
\end{frame}

\begin{frame}[c]{Common Features of CI systems}
  \begin{itemize}
    \item Build your application / library
    \item Run the test suite
    \item Do that for multiple OSes and software / compiler versions
    \item Build documentation and packages
    \item Upload and publish results / build products
  \end{itemize}
\end{frame}

\begin{frame}[c, fragile]{Example python workflow}
  \begin{center}\large
    \url{https://github.com/maxnoe/pyfibonacci/blob/main/.github/workflows/ci.yml}

    More details during the testing lecture.
  \end{center}
\end{frame}
